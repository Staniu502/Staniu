\documentclass[a4paper,12pt]{article}
\usepackage[MeX]{polski}
\usepackage[utf8]{inputenc}

%opening
\title{Wydział Matematyki i Informatyki Uniwersytetu
Warmińsko-Mazurskiego}
\author{Tomasz Stankiewicz}

\begin{document}
\maketitle
Wydział Matematyki i Informatyki Uniwersytetu Warmińsko-Mazurskiego (WMiI) – wydział
Uniwersytetu Warmińsko-Mazurskiego w Olsztynie oferujący studia na dwóch kierunkach:


\begin{abstract}

1 Misja

2 Opis kierunków

3 Struktura organizacyjna

4 Władze Wydziału

5 Historia Wydziału

6 Nowa siedziba Wydziału

7 Adres

8 Przypisy

9 Linki zewnętrzne



\end{abstract}

\section{Misja}
Misją Wydziału jest:

-Kształcenie matematyków zdolnych do udziału w rozwijaniu matematyki i jej stosowania w innych
działach wiedzy i w praktyce;

-Kształcenie nauczycieli matematyki, nauczycieli matematyki z fizyką a także nauczycieli informatyki;

-Kształcenie profesjonalnych informatyków dla potrzeb gospodarki, administracji, szkolnictwa oraz życia
społecznego;

-Nauczanie matematyki i jej działów specjalnych jak statystyka matematyczna, ekonometria,
biomatematyka, ekologia matematyczna, metody numeryczne; fizyki a w razie potrzeby i podstaw
informatyki na wszystkich wydziałach UWM.

\section{Opis kierunków}
Na kierunku Informatyka prowadzone są studia stacjonarne i niestacjonarne:
\ 

-Studia pierwszego stopnia – inżynierskie (7 sem.), sp. inżynieria systemów informatycznych, informatyka
ogólna
\


-Studia drugiego stopnia – magisterskie (4 sem.), sp. techniki multimedialne, projektowanie systemów
informatycznych i sieci komputerowych
\ 

\ 
Na kierunku Matematyka prowadzone są studia stacjonarne:
\ 
-Studia pierwszego stopnia – licencjackie (6 sem.), sp. nauczanie matematyki, matematyka stosowana


\ 
-Studia drugiego stopnia – magisterskie (4 sem.), sp. nauczanie matematyki, matematyka stosowana
oraz studia niestacjonarne:

\ 
-Studia drugiego stopnia – magisterskie (4 sem.), sp. nauczanie matematyki
\ 
\ 
\ 

Państwowa Komisja Akredytacyjna w dniu 19 marca 2009r. oceniła pozytywnie jakość kształcenia na kierunku
Matematyka, natomiast w dniu 12 marca 2015r. oceniła pozytywnie jakość kształcenia na kierunku
Informatyka [2]

\end{document}